\chapter{基本数学工具}

本附录包括了计量经济学\index{计量经济学}中用到的一些基本数学,我们扼要论述了求和算子的各种性质,研究了线性和某些非线性方程的性质,并复习了比例和百分数。我们还介绍了一些在应用计量经济学中常见的特殊函数,包括二次函数和自然对数,前 4 节只要求基本的代数技巧,第 5 节则对微分学进行了简要回顾;虽然要理解本书的大部分内容,微积分并非必需,但在一些章末附录和第 3 篇某些高深专题中,我们还是用到了微积分。

\section{求和算子与描述统计量}

\textbf{求和算子} 是用以表达多个数求和运算的一个缩略符号,它在统计学和计量经济学分析中扮演着重要作用。如果 $\{x_i: i=1, 2, \ldots, n\}$ 表示 $n$ 个数的一个序列,那么我们就把这 $n$ 个数的和写为:

\begin{equation}
\sum_{i=1}^n x_i \equiv x_1 + x_2 +\cdots + x_n
\end{equation}
\section{索引的生成方法}
\begin{enumerate}
	\item 在正文关键字处使用\lstinline|\index{}|命令,此命令类似于\lstinline|\label{}|命令
	\item 在你想要生成的索引的地方使用\lstinline|\printindex|命令
	\item 使用\hologo{XeLaTeX}编译一遍
	\item 在文档目录下打开终端,使用\lstinline|makeindex *.idx|命令\\(TeXstudio可以在工具栏选Tools-Index)
	\item 再到文档中使用\hologo{XeLaTeX}编译一遍即可
\end{enumerate}